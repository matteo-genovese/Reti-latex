\section{FAQ}
\begin{itemize}
    \item \textbf{IPv4 e frammentazione}
        \begin{itemize}
            \item I datagrammi IP vengono frammentati in datagrammi più piccoli quando un datagramma è più grande dell'MTU di un collegamento.
        \end{itemize}
    \item \textbf{DHCP}
        \begin{itemize}
            \item È una funzionalità di livello rete ma gestita da un processo a livello applicativo che utilizza delle socket UDP, porta con numero 67 (il server usa la porta 67 come \textit{porta di destinazione}), mentre i client aprono la porta 68 (il client usa la porta 68 come \textit{porta di destinazione}) con indirizzo IP $0.0.0.0$. DHCP è un \textit{protocollo a livello applicativo} che usa UDP.
            \item Il client appena inserito sulla rete non ha, giustamente, indirizzo IP e non conosce l'indirizzo IP del server DHCP, quindi il client invia un messaggio \textit{broadcast} (usando un \textit{indirizzo broadcast limitato} come 255.255.255.255) sulla porta 67, cercando un server DHCP, anche il server DHCP risponderà in \textit{broadcast} (usando un \textit{indirizzo broadcast limitato}), poiché il destinatario non ha ancora un \textit{indirizzo IP}.
            \item Consente di ottenere \textbf{dinamicamente} gli indirizzi IP degli \textit{host}, assegnando indirizzi IP da un \textit{pool di indirizzi disponibili}. Non assegna obbligatoriamente lo stesso IP all'host, varia in base a quelli che ha disponibile. Il \textbf{transaction ID} serve al \textit{server} per sapere con chi sta parlando e a chi assegnare l'\textit{indirizzo IP}, usato per \textit{abbinare richieste e risposte}.
        \end{itemize}
    \item \textbf{Come vengono memorizzati gli indirizzi del DNS}
        \begin{itemize}
            \item I server autoritativi sono i server che posseggono le traduzioni, sono di società che posseggono host Internet, devono fornire i record DNS di pubblico dominio che mappano i nomi di tali host in indirizzi IP. Sono la fonte finale dei record DNS per un dominio specifico.
            \item I server TLD sono responsabili dei domini ad alto livello (.com, .co.uk, .it...), vengono gestiti da nazioni o da aziende e sono responsabili di delegare l'autorità ai server autoritativi.
            \item I server radice sono il vertice della gerarchia e sono essenziali per il funzionamento del DNS. Sono 13 nel mondo (numero limitato), e verranno contattati da un DNS locale.
        \end{itemize}
    \item \textbf{Perché esiste IPv6}
        \begin{itemize}
            \item Il motivo per cui è stato progettato il protocollo \textbf{IP versione 6} è che si stanno esaurendo gli indirizzi IP a 32 bit. Gli indirizzi IPv6 sono \textit{128 bit}, rispetto ai 32 bit di IPv4. Inoltre è stato progettato per migliorare l'infrastruttura generale e velocizzarla.
        \end{itemize}
    \item \textbf{Struttura messaggi DNS}
        \begin{itemize}
            \item Formato RR (Record di risorsa): (name, value, type, ttl), dove \textit{ttl} rappresenta il \textit{time-to-live} del record. 4 tipi di type:
                \begin{itemize}
                    \item \textbf{Type=A}: name = nome host, value = indirizzo \textit{IPv4}.
                    \item \textbf{Type=NS}: name = dominio, value = nome del server di competenza, usato per \textit{delegare l'autorità} ad altri server DNS.
                    \item \textbf{Type=CNAME (canonical name)}: name = \textit{nome alias} di un nome \textit{nome canonico}, value = \textit{nome canonico}, usato per creare \textit{alias} per gli hostname.
                    \item \textbf{Type=MX}: value = nome del server di posta di \textit{name}, name = nome server, usato per \textit{instradare le email} al mail server appropriato.
                \end{itemize}
        \end{itemize}
    \item \textbf{FSM TCP}
        \begin{itemize}
            \item Il protocollo TCP offre vari servizi, tra cui: controllo di congestione e controllo di flusso, questo garantirà un trasferimento dati \textit{affidabile, ordinato e connection-oriented}, ma avrà ritardi nel trasporto a causa di un maggiore \textit{overhead} e \textit{potenziali ritardi}.
        \end{itemize}
    \item \textbf{DNS e struttura dei suoi messaggi}
        \begin{itemize}
            \item Il \textbf{DNS} è un protocollo, ma in realtà è un sistema *gerarchico* e *distribuito*, \textit{Domain Name System}.
            \item Il web è identificato da un nome simbolico, il nome dell'host, che è \textit{leggibile dagli umani}, ma il modo univoco per identificare il server è il suo indirizzo IP (32 bit), che è \textit{leggibile dalle macchine}.
            \item È un sistema distribuito, esistono vari server, organizzati in modo \textit{gerarchico}, che conoscono l'associazione fra nome host e indirizzo IP.
        \end{itemize}
    \item \textbf{NAT}
        \begin{itemize}
            \item Consente di separare una rete specifica dalle altre, crea la \textbf{rete privata}. L'acronimo sta per: \textbf{Network address translation}, usato per \textit{conservare gli indirizzi IP pubblici}.
            \item Il \textbf{NAT} si trova all'interno del \textit{router}, che si trova al \textit{confine} tra la rete privata e quella pubblica. Le reti esterne vedono la rete privata come un \textbf{unico} \textit{indirizzo IP}, che sarà quello assegnato al \textit{router}. L'indirizzo IP pubblico è l'\textit{unico indirizzo IP} visibile all'esterno. Le macchine all'interno della \textit{rete privata} avranno degli \textbf{indirizzi IP privati}, che saranno visibili solo all'interno della \textit{rete privata}. Le reti private usano \textit{range di indirizzi IP privati}.
            \item Il router NAT \textit{mantiene una tabella di traduzione} per mappare indirizzi IP privati e porte a indirizzi IP pubblici e porte.
        \end{itemize}
    \item \textbf{Struttura pacchetto TCP e cosa è}
        \begin{itemize}
            \item Il protocollo TCP offre vari servizi, tra cui: controllo di congestione e controllo di flusso, questo garantirà un trasferimento dati \textit{affidabile, ordinato e connection-oriented}, ma avrà ritardi nel trasporto a causa di un maggiore \textit{overhead} e \textit{potenziali ritardi}.
            \item  Dopo la fase di \textit{handshaking} si dice che esiste una \textbf{connessione TCP} tra le socket di client e host, una connessione di tipo \textit{full-duplex} (i processi possono scambiare messaggi contemporaneamente in entrambe le direzioni).
        \end{itemize}
    \item \textbf{Cosa sono le socket}
        \begin{itemize}
            \item Le \textbf{socket} mettono in comunicazione \textit{livello applicativo} con il \textit{livello di trasporto}. Le socket sono \textit{endpoint} per la comunicazione. Esistono due tipi di \textit{socket}: \textbf{UDP} e \textbf{TCP}.
        \end{itemize}
    \item \textbf{RTT di TCP}
        \begin{itemize}
            \item \textbf{Round-trip time (RTT)}: tempo impiegato da un pacchetto per viaggiare dal client al server e tornare al client. È una \textit{misura} della latenza di rete. Include i ritardi di propagazione, di accodamento nei router e nei commutatori intermedi e di elaborazione del pacchetto.
        \end{itemize}
    \item \textbf{Come è fatto un segmento UDP}
        \begin{itemize}
            \item L'intestazione UDP è formata da numero porta origine, numero porta di destinazione, lunghezza in byte del segmento UDP con intestazione e il checksum (che è \textit{opzionale} in IPv4, aggiunge bit alla fine per controllare se viene corrotto il pacchetto), tutti tasselli da 16 bit, totale di 8 Byte.
        \end{itemize}
    \item \textbf{Rappresentazione grafica RTT ed stima RTT}
        \begin{itemize}
            \item Il problema principale è stimare correttamente la durata del timer utilizzando la \textbf{media mobile esponenziale ponderata}. L'obiettivo è impostare un timeout \textit{ragionevole}, che non sia troppo corto (causando ritrasmissioni inutili) né troppo lungo (causando ritardi). La media mobile esponenziale ponderata è una tecnica che dà \textit{più peso ai campioni recenti} del RTT.
        \end{itemize}
    \item \textbf{Messaggi posta elettronica}
        \begin{itemize}
            \item L'\textit{user agent} (un'applicazione usata per comporre, inviare e ricevere email) invia il messaggio al proprio \textit{mail server} (il distributore del servizio che memorizza e inoltra i messaggi) che invierà la mail al \textit{mail server} del destinatario.
            \item Componenti della posta elettronica:
                \begin{itemize}
                    \item \textbf{casella di posta}: contenitore dei messaggi in arrivo, collocata in un \textit{mail server}. Ogni utente ha una casella di posta \textit{unica} su un mail server.
                    \item \textbf{Coda di messaggi}: mail che devono arrivare al destinatario, memorizzate \textit{temporaneamente} in attesa di consegna.
                    \item \textbf{Protocollo SMTP} (Simple Mail Transfer Protocol): regola la comunicazione tra i \textit{mail server} e tra gli \textit{user agent} e i proprio \textit{mail server}, usato per \textit{trasferire} i messaggi. Principale protocollo a livello di applicazione per la posta elettronica, utilizza \textit{TCP}.
                \end{itemize}
        \end{itemize}
    \item \textbf{Congestione Tcp}
        \begin{itemize}
            \item Per le leggi fisiche non è possibile scambiare dati istantaneamente. Le reti introducono ritardi, perdono pacchetti e limitano il \textbf{throughput} (la quantità di dati al secondo che può essere trasferita tra due sistemi periferici) a causa di fattori come la congestione della rete e le limitazioni fisiche dei collegamenti.
        \end{itemize}
    \item \textbf{Controllo di flusso}
        \begin{itemize}
            \item Il controllo di flusso serve a \textit{prevenire che il mittente sovraccarichi il ricevitore}.
            \item Il valore di \textbf{RcvWindow} verrà inserito all'interno dei segmenti, il mittente usa il valore di `RcvWindow` per \textit{limitare la quantità di dati non riscontrati} che invia, così che non vengano inviati dati che verranno sicuramente persi. Il campo `RcvWindow` indica la \textit{quantità di spazio libero} nel buffer del ricevitore.
        \end{itemize}
    \item \textbf{Multiplexing e demultiplexing tcp}
        \begin{itemize}
            \item Il demultiplexing TCP si basa sull'identificazione delle socket TCP tramite una quaterna: indirizzo IP di origine, numero di porta di origine, indirizzo IP di destinazione e numero di porta di destinazione. Questa quaterna identifica \textit{univocamente una connessione TCP}.
        \end{itemize}
    \item \textbf{Tabella di inoltro di un router}
        \begin{itemize}
            \item Le tabelle di inoltro, utilizzate per determinare la porta di uscita di un pacchetto, sono elaborate e aggiornate dal processore di instradamento. Una copia di queste tabelle è memorizzata su ciascuna porta di ingresso per velocizzare il processo di inoltro.
            \item La ricerca nella tabella di inoltro si basa sul confronto tra un prefisso dell'indirizzo di destinazione del pacchetto e le righe della tabella.
            \item Se un indirizzo di destinazione corrisponde a più righe, il router adotta la regola di corrispondenza a prefisso più lungo, inoltrando il pacchetto all'interfaccia di collegamento associata alla corrispondenza più lunga.
        \end{itemize}
    \item \textbf{Throughput tcp}
        \begin{itemize}
            \item Ogni pacchetto viene trasmesso sul collegamento fisico alla velocità di trasmissione $R$ (misurata in bit al secondo, bps). Un pacchetto di $L$ bit impiegherà quindi $L/R$ secondi per essere trasmesso completamente sul collegamento.
        \end{itemize}
    \item \textbf{Equità tcp}
        \begin{itemize}
            \item TCP, attraverso il meccanismo di controllo della congestione, cerca di fornire una condivisione "\textbf{equa}" della banda tra le connessioni. In uno scenario ideale, le connessioni TCP che condividono un collo di bottiglia dovrebbero ottenere un throughput di circa $R/K$, dove $R$ è la capacità del collegamento e $K$ è il numero di connessioni. L'algoritmo AIMD contribuisce alla \textbf{fairness}, aumentando la finestra di congestione linearmente e diminuendola moltiplicativamente in caso di perdita.
        \end{itemize}
    \item \textbf{Ipv6}
        \begin{itemize}
            \item Il motivo per cui è stato progettato il protocollo \textbf{IP versione 6} è che si stanno esaurendo gli indirizzi IP a 32 bit. Gli indirizzi IPv6 sono \textit{128 bit}, rispetto ai 32 bit di IPv4. Inoltre è stato progettato per migliorare l'infrastruttura generale e velocizzarla.
        \end{itemize}
\end{itemize}
