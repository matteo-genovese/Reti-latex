\section{Reti di calcolatori e Internet}

\subsection{Cos'è internet?}
\textbf{Internet} è una rete di calcolatori connesse tra di loro

\subsubsection{Definizioni varie}
\begin{description}[font=\sffamily\bfseries, leftmargin=1cm, style=nextline]
  \item[host] 
    \textit{o sistemi periferici} sistema terminale della rete, coloui che svolge operazioni.
  \item[rete di collegamenti]
    o \textit{communication link} è un tipo di connessione per collegare gli host.
  \item[commutatori di pacchetti]
    o \textit{packet switch} è un altro tipo di connessione per collegare gli host.
  \item[velocità di trasmissione]
    o \textit{transimission rate} velocità con cui i vari tipi di collegamenti si scambiano dati, misurata in \textbf{bit/secondo (bps)}.
  \item[pacchetto]
    \textit{o packet} divisione del dato in sottoparti (\textbf{segmento}) che gli host si devono scambiare con l'aggiunta di intestazione.
  \item[asd]
\end{description}

\subsection{Ai confini della rete}
\subsubsection{Le reti di accesso}
\subsubsection{Mezzi trasmissivi}

\subsection{Il nucleo della rete - importante}
\subsubsection{Commutazione di pacchetto}
\subsubsection{Commutazione di circuito}
\subsubsection{Una rete di reti}

\subsection{Ritardi, perdite e throughput nelle reti a commutazione di pacchetto - importante}
\subsubsection{Panoramica del ritardo nelle reti a commutazione di pacchetto}
\subsubsection{Ritardo di accodamento e perdita di pacchetti}
\subsubsection{Ritardo end-to-end} 
\subsubsection{Throughput nelle reti di calcolatori} 

\subsection{Livelli dei protocolli e loro modelli di servizio - importante} 
\subsubsection{Architettura a livelli }
\subsubsection{Incapsulamento}

