\section{Livello di applicazione}

\subsection{Princìpi delle applicazioni di rete}

\subsubsection{Architetture delle applicazioni di rete}
Lo sviluppatore di applicazioni deve progettare l'\textbf{architettura dell'applicazione} (\textit{application architecture}). Ci sono due tipi di architettura: \textbf{client-server} o \textbf{P2P (peer-to-peer)}.

L'\textbf{architettura client-server} si basa su un \textit{server} che sarà sempre attivo, con un indirizzo IP statico e sarà connesso con altri server, mentre il \textit{client} è colui che \textbf{inizia} la comunicazione con il server. Non succederà mai che il server proverà a contattare il client, e i client avranno degli indirizzi IP dinamici. Costi alti per via dell'installazione e manutenzione. Nel caso in cui il server cada e non sarà più raggiungibile tutta la rete cadrà, quindi abbiamo un \textbf{singolo punto di fallimento} che è il server.

L'\textbf{architettura P2P pura} si basa sulla comunicazione tra i vari \textit{peer}, dove ogni peer agisce sia come \textit{client} che come \textit{server}. Non esiste un \textit{host} sempre attivo. È un'architettura scalabile, poiché le risorse sono distribuite tra i vari peer, ma difficile da gestire per la sicurezza (tutti gli host devono essere protetti adeguatamente, se uno solo non è protetto tutta la rete è insicura) visto che manca un punto centrale di controllo, e ogni peer è un potenziale punto di fallimento.

Esiste un'architettura \textbf{ibrida}, quindi un mix di architettura client-server e P2P. Il server serve come mezzo di ricerca, fungendo da directory o tracker. I peer mandano una richiesta al server che la inoltra agli altri peer, mettendo poi in comunicazione i peer nella modalità P2P.

\subsubsection{Processi comunicanti}
Nei sistemi operativi i programmi prendono nome di \textbf{processi comunicanti}.
\begin{itemize}
    \item \textbf{Processo}: programma in esecuzione su un host. Più processi sullo stesso host comunicano tramite \textbf{schemi interprocesso}, mentre processi su host differenti comunicano tramite scambio di messaggi.
    \item \textbf{Processo client}: processo che dà inizio alla comunicazione.
    \item \textbf{Processo server}: processo che attende di essere contattato.
    \item \textbf{Socket}: Il processo comunica tramite un socket, un'interfaccia che mette in comunicazione il livello del processo applicativo e il livello trasporto. È equiparabile a una porta logica.
    \item \textbf{API}: Application Programming Interface, definisce come un'applicazione accede ai servizi di trasporto. È un'interfaccia di programmazione.
\end{itemize}

Le applicazioni con architettura P2P hanno sia processi client che server. Il progettista dell'applicazione può scegliere il protocollo di trasporto e alcuni parametri a livello di trasporto.

Per identificare il processo ricevente bisogna avere due informazioni:
\begin{itemize}
  \item \textbf{Indirizzo dell'host}: specificato dal loro \textbf{indirizzo IP}, un indirizzo logico di 32 bit che identifica univocamente l'host.
  \item \textbf{Identificatore del processo ricevente sull'host di destinazione}: la sua \textbf{socket}, il \textbf{numero di porta di destinazione} svolge questo compito, identificando un processo specifico sull'host.
\end{itemize}

\includegraphics[width=\textwidth, height=6cm, keepaspectratio]{img/processi-socket.png}

\subsubsection{Servizi di trasporto disponibili per le applicazioni}
L'applicazione lato mittente trasmette i messaggi tramite la socket. Lato ricevente, il protocollo a livello di trasporto deve consegnare i messaggi alla socket del processo ricevente, fornendo una \textit{comunicazione logica} tra le applicazioni.

Esistono vari servizi offerti dai protocolli a livello di trasporto, tra cui: trasferimento dati affidabile, throughput, temporizzazione e sicurezza.

\subsubsection*{Trasferimento dati affidabile}
Molte applicazioni hanno la necessità di ricevere ogni pacchetto che gli viene mandato, poiché la perdita di pacchetto potrebbe causargli dei danni, per esempio economici. Hanno bisogno di un protocollo con un servizio di consegna garantita dei dati, cioè il \textbf{trasferimento affidabile dei dati} (\textit{reliable data transfer}), che si ottiene tramite meccanismi come acknowledgment e ritrasmissione.

Se il protocollo a livello di trasporto fornisce questo servizio, il processo mittente manderà i dati sapendo che arriveranno tutti a destinazione.

Alcune applicazioni, per esempio quelle multimediali, accettano la possibilità di perdita di dati poiché preferiscono la maggiore velocità di trasmissione, si dicono \textbf{applicazioni che tollerano le perdite} (\textit{loss-tolerant applications}), ovvero applicazioni che possono tollerare una certa perdita di dati senza impatti significativi sulla loro funzionalità.

\subsubsection*{Throughput}
Garantire il servizio di \textbf{throughput} significa garantire una certa quantità di banda per il collegamento, ovvero una certa \textbf{velocità} con cui i dati vengono trasferiti. Queste applicazioni vengono chiamate \textbf{applicazioni sensibili alla banda} (\textit{bandwidth-sensible applications}), ovvero applicazioni che richiedono un \textbf{throughput minimo garantito}.

Esistono delle applicazioni che non hanno bisogno di una quantità di throughput garantita, ma sono elastiche, da qui il nome \textbf{applicazioni elastiche} (\textit{elastic applications}), ovvero applicazioni che si adattano al throughput disponibile in quell'istante.

\subsubsection*{Temporizzazione}
Per le applicazioni in tempo reale è necessaria una comunicazione quasi istantanea, hanno bisogno della garanzia che la trasmissione avvenga sotto un determinato tempo. Queste garanzie di temporizzazione sono spesso difficili da ottenere nella pratica.

\subsubsection*{Sicurezza}
Infine, un protocollo a livello di trasporto può garantire la sicurezza dei dati mediante, per esempio, la crittografia. I servizi di sicurezza possono includere confidenzialità, integrità e autenticazione.

\subsubsection{Servizi di trasporto offerti da Internet}
Esistono due protocolli di trasporto per le applicazioni di internet:
\begin{itemize}
  \item \textbf{TCP}: prevede una connessione e un trasporto affidabile dei dati.
  \textbf{Servizio orientato alla connessione} \textit{(connection-oriented service)}: fa in modo che client e server si scambino informazioni di controllo a livello di trasporto prima che i messaggi a livello di applicazione comincino a fluire. Questa procedura, denominata \textbf{handshaking}, stabilisce uno \textit{stato} sia sul client che sul server, pre-allertandoli per lo scambio di messaggi.
    Dopo la fase di \textit{handshaking} si dice che esiste una \textbf{connessione TCP} tra le socket di client e host, una connessione di tipo \textit{full-duplex} (i processi possono scambiare messaggi contemporaneamente in entrambe le direzioni).
    \textbf{Servizio di trasferimento affidabile} \textit{(reliable data transfer service)}: il protocollo TCP garantisce ai processi il flusso di dati senza errori, utilizzando meccanismi come acknowledgment, ritrasmissione e numeri di sequenza.
    TCP evita la congestione della rete, \textbf{attivamente} strozzando il flusso quando è eccessivo.
  \item \textbf{UDP}: protocollo minimale, è "senza connessione" (non ha bisogno di \textit{handshaking}), ciò non garantisce l'affidabilità della connessione. UDP non garantisce la consegna, l'ordine o l'assenza di duplicati. Non ha un meccanismo di gestione della congestione, manda il flusso di dati al livello di rete a qualsiasi velocità, lasciando la gestione della congestione all'applicazione.
\end{itemize}

\subsubsection{Protocolli a livello di applicazione}
I protocolli a livello di applicazione devono stabilire:
\begin{itemize}
\item i tipi di messaggi scambiati (richiesta o risposta), definendo lo \textbf{scopo} del messaggio.
\item la sintassi dei vari tipi di messaggio, definendo il \textbf{formato} del messaggio.
\item la semantica dei campi, definendo il \textbf{significato} dei campi all'interno del messaggio.
\item le regole per regolare quando e come un processo invia e risponde ai messaggi, definendo la \textbf{sequenza} dei messaggi e le \textbf{azioni} intraprese dai processi comunicanti.
\end{itemize}

\subsection{Web e HTTP}
Il \textbf{Web} è \textit{on demand}, ciò significa che si può avere quello che si vuole quando si vuole, accedendo a risorse tramite \textbf{URL} \textit{(Uniform Resource Locator)}.

\subsubsection{Panoramica di HTTP}
\textbf{HTTP} (\textit{HyperText Transfer Protocol}) è un protocollo a livello di applicazione \textit{request-response} che utilizza \textbf{TCP} come protocollo di trasporto, implementato a due programmi, client e server.
Una \textbf{pagina web} è un documento costituito da \textit{più} oggetti, un \textbf{oggetto} è un file indirizzabile tramite un \textbf{URL}.
In un \textit{URL} ci sono tre componenti: il protocollo (es. `http://`), il nome dell'host e il percorso dell'oggetto.
I \textbf{browser} implementano il protocollo \textit{HTTP} lato client, \textit{richiedendo} oggetti.
I \textbf{server web} implementano il protocollo \textit{HTTP} lato server (Apache è un esempio), \textit{fornendo} oggetti e saranno sempre attivi con un indirizzo IP statico.
\textit{HTTP} è un protocollo \textbf{senza memoria di stato} (\textit{stateless protocol}), ovvero i server HTTP non mantengono alcuna informazione sulle richieste passate dei client.

\subsubsection{Connesioni persistenti e non persistenti}
Si possono creare due tipi di connessioni:
\begin{itemize}
  \item \textbf{non persistenti}: ogni coppia richiesta e risposta deve essere inviata su una connessione TCP \textit{separata}. C'è un overhead nell'aprire una nuova connessione TCP per ogni coppia richiesta-risposta.
  \item \textbf{persistenti}: tutte le comunicazioni sono mandate sulla stessa connessione TCP (scelta di default per HTTP), riutilizzando la stessa connessione TCP e riducendo l'overhead.
\end{itemize}

\subsubsection*{HTTP con connessioni non persistenti}
Nelle \textit{connessioni non persistenti TCP} ogni connessione viene chiusa dopo l'invio dell'oggetto da parte del server, quindi ogni connessione trasporterà un solo messaggio di richiesta e solo uno di risposta.
Esistono browser che possono aprire più connessioni TCP parallelamente per velocizzare.
Il \textbf{round-trip time (RTT)} è il tempo impiegato da un pacchetto per viaggiare dal client al server e tornare al client. È una \textit{misura} della latenza di rete. Include i ritardi di propagazione, di accodamento nei router e nei commutatori intermedi e di elaborazione del pacchetto. \\ 
\includegraphics[width=\textwidth]{./img/rtt.png}
Quando un utente clicca su un collegamento ipertestuale il browser inizializza una connessione TCP con il web server, iniziando così un \textbf{handshake a tre vie} (three-way handshake): il client invia un piccolo segmento TCP al server (SYN), il server manda una conferma sempre con un piccolo segmento TCP (SYN-ACK), e il client manda una conferma di ritorno al server (ACK). Questo handshake serve a stabilire la connessione TCP.
Con queste prime due operazioni di handshake a tre vie calcoliamo il \textit{RTT}.
Il client ora invia un messaggio di richiesta HTTP insieme alla conferma di avvenuta ricezione (\textit{acknowledgement - ACK}), a messaggio ricevuto dal server il server procede a inviare il file al client. Il tempo di risposta totale sarà di 2 RTT (un RTT per l'handshake TCP e un RTT per la richiesta-risposta HTTP) più il tempo di trasmissione del file dal server al client.

\subsubsection*{HTTP con connessioni persistenti}
Nelle connessioni persistenti il server lascia aperta la connessione dopo la prima coppia di richiesta-risposta col client, tutte le altre coppie verranno trasmesse sulla stessa connessione, riutilizzando la stessa connessione TCP e riducendo l'overhead.
Una delle caratteristiche di queste connessioni è il \textit{pipelining}, la capacità di poter effettuare \textit{più} richieste senza aspettare la risposta delle richieste in corso.
La connessione si chiuderà dopo un lasso di tempo configurato in cui la connessione è stata inattiva.

\subsubsection{Formato dei messaggi HTTP}
Esistono due formati di messaggi HTTP, basati su testo, per richiesta e risposta.

\subsubsection*{Messaggio di richiesta HTTP}
\begin{quote}
  GET /somedir/page.html HTTP/1.1 \newline
  Host: www.someurl.com \newline
  Connection: close \newline
  User-agent: Mozilla/5.0 \newline
  Accept-language: fr
\end{quote}

Questa è una richiesta HTTP, può avere un numero indefinito di righe, la riga fondamentale è la prima che è la \textbf{riga di richiesta}, le successive sono \textbf{righe di intestazione}.

La riga di richiesta ha 3 campi: metodo (GET, POST, DATA, PUT, DELETE), che specifica l'\textit{azione} da eseguire sulla risorsa, l'URL e la versione di HTTP.
\textbf{GET} è il metodo più utilizzato nel web, si usa per \textit{recuperare} un oggetto tramite l'URL.

Notiamo la riga "Connection: close", con questa riga il browser comunica al server che non si deve occupare di connessioni persistenti, deve chiudere la connessione dopo aver inviato l'oggetto. Questo header è tipico delle connessioni non persistenti.

Alla fine della richiesta troviamo un \textit{corpo} del messaggio, vuoto in caso di metodo GET, utilizzato in caso di metodo POST per inviare dati al server (es. dati di un form).

Il metodo \textbf{POST} viene utilizzato per \textit{inviare} form compilati dall'utente al server, si può utilizzare anche il metodo GET per questo scopo ma includendo questi dati nell'URL della pagina richiesta.

Il metodo \textbf{HEAD} viene utilizzato dagli sviluppatori, è come il metodo \textit{GET} ma si riceve solo la risposta HTTP, senza ricevere l'oggetto, quindi si recuperano i \textit{metadata} di una risorsa.

Il metodo \textbf{PUT} viene utilizzato per caricare dal client dei file sul server.

Il metodo \textbf{DELETE} viene utilizzato per cancellare file sul server (spesso disabilitato per ragioni di sicurezza insieme al metodo PUT).

\includegraphics[width=\textwidth]{./img/richiestaHTTP.png}

\subsubsection*{Messaggio di risposta HTTP}
\begin{quote}
  HTTP/1.1 200 OK
  Connection: close \newline
  Date: Thu, 18 Aug 2015 15:44:04 GMT \newline
  Server: Apache/2.2.3 (CentOS) \newline
  Last-Modified: Tue, 18 Aug 2015 15:11:03 GMT \newline
  Content-Lenght: 6821 \newline
  Content-Type: text/html \newline
  (data data data data ....)
\end{quote}

Abbiamo: una \textbf{riga di stato} iniziale (contenente 3 campi, versione di protocollo, codice di stato HTTP che indica l'\textit{esito} della richiesta e il corrispettivo messaggio), sei \textbf{righe di intestazione} e il \textbf{corpo dell'oggetto} finale (fulcro del messaggio, contiene l'oggetto richiesto).

Esistono vari codici di stato, questi i più comuni:
\begin{itemize}
  \item \textbf{200 - OK}: la richiesta ha avuto successo e in risposta si invia l’informazione.
  \item \textbf{301 - Moved Permanently}: l’oggetto richiesto è stato trasferito in modo permanente; il nuovo URL è specificato nell’intestazione Location: del messaggio di risposta. Il client recupererà automaticamente il nuovo URL.
  \item \textbf{400 - Bad Request}: si tratta di un codice di errore generico che indica che la richiesta non è stata compresa dal server.
  \item \textbf{404 - Not Found}: il documento richiesto non esiste sul server.
  \item \textbf{505 - HTTP Version Not Supported}: il server non dispone della versione di protocollo HTTP richiesta.
\end{itemize}

\includegraphics[width=\textwidth]{./img/rispostahttp.png}

\subsubsection{Cookie}
HTTP è un protocollo \textit{stateless}, un elemento utile per i webserver sono i \textbf{cookie}, un identificativo per l'utente che mantiene le informazioni sul server, per esempio l'\textit{autenticazione}, e che permette di \textit{mantenere lo stato} tra più richieste HTTP. 

È formato da 4 componenti, tra cui:
\begin{itemize}
  \item Riga di intestazione nel messaggio di risposta HTTP (\textbf{Set-cookie}: numero identificativo), usata per \textit{impostare} il cookie.
  \item Riga di intestazione nel messaggio di richiesta HTTP (\textbf{Cookie}: numero identificativo), usata per \textit{inviare} il cookie al server.
  \item File cookie, mantenuto \textit{localmente} sul sistema terminale dell'utente e gestito dal browser del client.
  \item Database sul webserver che mantiene l'identificativo dei cookie, usato per \textit{associare} i cookie con i dati dell'utente.
\end{itemize}

I cookie possono anche essere usati per creare un livello di sessione utente al di sopra di HTTP che è privo di stato, permettendo di mantenere informazioni sull'utente durante la sua navigazione.

\subsubsection{Web caching (proxy server)}

Una \textbf{web cache}, cioè un \textbf{proxy server}, agisce come \textit{intermediario} e gestisce le richieste HTTP al posto del web server effettivo. All'interno del proxy server troviamo delle \textit{copie} dei siti già richiesti da altri web server, si trovano all'interno della sua cache.

Il meccanismo di funzionamento è:
\begin{enumerate}
  \item Il browser stabilisce una connessione TCP con il proxy server e invia una richiesta HTTP per l'oggetto specificato.
  \item Abbiamo due casi possibili ora:
    \begin{enumerate}
    \item Il proxy server ha l'oggetto richiesto nella propria cache, inoltra direttamente l'oggetto al browser che ne ha fatto richiesta.
    \item Il proxy server non ha l'oggetto richiesto, apre una connessione TCP verso il server in cui risiede l'oggetto richiesto, effettua una richiesta HTTP al server e poi inoltra l'oggetto al browser.
    \end{enumerate}
  \item Il proxy riceve l'oggetto e salva la copia sulla sua cache.
\end{enumerate}

Notiamo che il proxy server è contemporaneamente sia \textit{client} (quando richiede oggetti al server di origine) che \textit{server} (quando fornisce oggetti al client). Il suo utilizzo riduce notevolmente i tempi di caricamento dei siti, il traffico di rete e il carico del server.

Uno dei problemi che si possono venire a creare è che l'oggetto richiesto al proxy server potrebbe non essere aggiornato, portando a visualizzare informazioni obsolete. Per ovviare a questo problema è stato implementato il \textbf{GET condizionale} (\textit{conditional GET}), è come una richiesta GET HTTP tradizionale, in più si aggiunge una riga di intestazione \texttt{if-modified-since:}, che include un \textit{timestamp}. Se il server riceve una richiesta GET condizionale, confronta il timestamp nell'header `if-modified-since` con il timestamp dell'ultima modifica dell'oggetto. Se l'oggetto non è stato modificato dopo il timestamp specificato, il server risponde con un codice di stato \textbf{304 Not Modified} senza inviare l'oggetto. Se l'oggetto è stato modificato, il server risponde con un codice di stato \textbf{200 OK} e invia l'oggetto aggiornato.

\subsection{Posta elettronica}
La posta elettronica è un mezzo di comunicazione asincrono, ciò significa che non richiede che mittente e destinatario siano online contemporaneamente, tre componenti principali: gli \textbf{user agent} (o agenti utente), i \textbf{mail server} (server di posta) e il \textbf{protocollo SMTP (Simple Mail Transfer Protocol)}.
L'\textit{user agent} (un'applicazione usata per comporre, inviare e ricevere email) invia il messaggio al proprio \textit{mail server} (il distributore del servizio che memorizza e inoltra i messaggi) che invierà la mail al \textit{mail server} del destinatario.
Componenti della posta elettronica:
\begin{itemize}
  \item \textbf{casella di posta}: contenitore dei messaggi in arrivo, collocata in un \textit{mail server}. Ogni utente ha una casella di posta \textit{unica} su un mail server.
  \item \textbf{Coda di messaggi}: mail che devono arrivare al destinatario, memorizzate \textit{temporaneamente} in attesa di consegna.
  \item \textbf{Protocollo SMTP} (Simple Mail Transfer Protocol): regola la comunicazione tra i \textit{mail server} e tra gli \textit{user agent} e i proprio \textit{mail server}, usato per \textit{trasferire} i messaggi. Principale protocollo a livello di applicazione per la posta elettronica, utilizza \textit{TCP}.
    Quando un server invia posta a un altro, agisce come client SMTP; quando invece la riceve, funziona come server SMTP. Il protocollo SMTP regola la comunicazione tra i \textit{mail server} e tra gli \textit{user agent} e i proprio \textit{mail server}.
\end{itemize}

\subsubsection{SMTP}
Il protocollo \textbf{SMTP} utilizza TCP, la porta standard è la $25$, è un \textit{protocollo con stato}, ciò significa che mantiene lo stato \textit{tra più comandi} all'interno della stessa connessione, codificato in ASCII-7bit per i \textit{messaggi di controllo}, mentre il corpo del messaggio può usare altre codifiche.

Si prevedono 3 fasi diverse:
\begin{itemize}
  \item Prima fase: handshake, che include lo scambio di comandi specifici come \texttt{HELO} o \texttt{EHLO}.
  \item Seconda fase: scambio effettivo dei messaggi, che include comandi come \texttt{MAIL FROM} (per specificare il mittente), \texttt{RCPT TO} (per specificare il destinatario) e \texttt{DATA} (per inviare il corpo del messaggio).
  \item Terza fase: chiusura della connessione, che include il comando \texttt{QUIT}.
\end{itemize}

\subsubsection*{Scenario di funzionamento del protocollo}
\begin{enumerate}
  \item L'user agent mittente compone l'indirizzo mail del destinatario.
  \item L'user agent invia il messaggio al mail server del mittente.
  \item Il mail server mittente apre una connessione TCP con il mail server del destinatario sulla porta 25, agendo come client SMTP.
  \item Il client SMTP (mail server mittente) invia il messaggio tramite la connessione TCP e lo colloca nella coda dei messaggi.
  \item Il mail server del destinatario, agendo come server SMTP, riceve il messaggio e lo invia nella mail box.
  \item L'user agent destinatario può ora leggere il messaggio.
\end{enumerate}

\includegraphics[width=\textwidth]{./img/smtp.png}

Si utilizzano dei mail server poiché nel caso in cui le mail non possono essere consegnate in un preciso momento o ci sono errori, il mail server può fare vari tentativi (in base alla configurazione fatta) per consegnare il messaggio, utilizzando una \textit{coda} per memorizzare i messaggi e ritentare la consegna in un secondo momento.

Oltre ai comandi di controllo, SMTP definisce anche il formato del messaggio email, che include header (es. From, To, Subject) e un corpo. La fine del messaggio email è indicata da un punto su una riga vuota. Per recuperare le email dalla casella di posta, gli user agent utilizzano protocolli come \textbf{POP3} (Post Office Protocol version 3), che permette di scaricare le email dal server e, opzionalmente, di cancellarle dal server (download-and-delete) o di mantenerle (download-and-keep), o \textbf{IMAP} (Internet Message Access Protocol), che permette di gestire le mailbox direttamente sul server, consentendo di visualizzare le email, creare cartelle, spostare messaggi e altre operazioni senza dover scaricare tutte le email sul dispositivo locale.

\subsection{DNS}
Il \textbf{DNS} è un protocollo, ma in realtà è un sistema gerarchico e distribuito, \textit{Domain Name System}.
Il web è identificato da un nome simbolico, il nome dell'host, che è \textit{leggibile dagli umani}, ma il modo univoco per identificare il server è il suo indirizzo IP (32 bit), che è \textit{leggibile dalle macchine}.
È un sistema distribuito, esistono vari server, organizzati in modo \textit{gerarchico}, che conoscono l'associazione fra nome host e indirizzo IP.
È un protocollo a livello di applicazione, i vari servizi che offre sono:
\begin{itemize}
  \item Traduzione degli hostname in indirizzi IP, che è la funzione \textit{principale} del DNS.
  \item Host aliasing, più nomi per lo stesso indirizzo IP (stesso host). Esiste un \textbf{nome canonico} della macchina che la identifica, esistono anche degli \textbf{alias} che sono altri nomi per la stessa macchina, usati per \textit{convenienza} e \textit{flessibilità}.
  \item Possibilità di gestire la posta elettronica in server diversi ma con lo stesso hostname (esempio mail@unipa.it e unipa.it), usando i record DNS per \textit{mappare} i nomi di dominio ai mail server.
  \item Possibilità di distribuire il carico, più macchine possono gestire lo stesso server, ci saranno più server (e più indirizzi IP) per un unico hostname, e DNS può distribuire il traffico su \textit{più server}.
\end{itemize}

\subsubsection{Gestione gerarchica DNS}
Abbiamo un sistema gerarchico, progettato per \textit{distribuire il carico} e \textit{migliorare la scalabilità}, più in alto abbiamo i \textbf{server radice}, sotto i \textbf{server TLD (top-level domain)} e infine i server \textbf{autoritativi (o di competenza)}.

I \textit{server autoritativi} sono i server che posseggono le traduzioni, sono di società che posseggono host Internet, devono fornire i record DNS di pubblico dominio che mappano i nomi di tali host in indirizzi IP. Sono la \textit{fonte finale} dei record DNS per un dominio specifico.

I \textit{server TLD} sono responsabili dei domini ad alto livello (.com, .co.uk, .it\dots), vengono gestiti da nazioni o da aziende e sono responsabili di \textit{delegare l'autorità} ai server autoritativi.

I \textit{server radice} sono il vertice della gerarchia e sono \textit{essenziali} per il funzionamento del DNS. Sono 13 nel mondo (numero limitato), e verranno contattati da un \textbf{DNS locale}.

Il client contatterà sempre il server radice, che darà indicazioni su dove trovare i server TLD interessati che diranno al client qual è il server autoritativo responsabile per l'hostname scelto. In realtà, il client contatta un \textbf{DNS locale} (o \textit{ricorsivo}) che poi effettua le query iterative o ricorsive.

\subsubsection{DNS locale}
Una macchina fuori dalla gerarchia, che si occuperà di fare da client per le comunicazioni tra il client reale e l'host radice, funzionando da intermediario. I DNS locali agiscono come \textit{resolver ricorsivi}, effettuando le query per conto dei client.

Ogni ISP ha un \textbf{DNS locale} e lui opera da proxy, inoltra la query in una gerarchia di server DNS. Gli ISP forniscono DNS locali ai loro clienti. I DNS locali fanno \textit{caching} dei record DNS per migliorare le prestazioni.

Approccio con \textbf{expiring date} (o TTL - Time-To-Live) per il mantenimento delle informazioni, usato per \textit{invalidare} i record in cache.

\subsubsection*{Esempio di query iterativa}
\begin{enumerate}
\item Il \textbf{client web richiedente} chiede al \textbf{client DNS proprio} (che è tipicamente parte del sistema operativo) di tradurre un hostname.
\item Il \textbf{client DNS} parla col \textbf{server DNS locale}.
\item Il \textbf{server DNS locale} richiede informazioni al \textbf{server DNS radice}.
\item Il \textbf{server DNS radice} fornisce un riferimento, al \textbf{server TLD}, al \textbf{server DNS locale}.
\item Il \textbf{server DNS locale} richiede informazioni al \textbf{server DNS TLD}.
\item Il \textbf{server DNS TLD} fornisce un riferimento, al \textbf{server di competenza}, al \textbf{server DNS locale}.
\item Il \textbf{server DNS locale} richiede informazioni al \textbf{server di competenza}.
\item Il \textbf{server di competenza} fornisce l'IP corrispondente dell'hostname al \textbf{server DNS locale}.
\item Il \textbf{server DNS locale}, ricevendo l'informazione dal \textbf{server di competenza}, la manda al \textbf{client dns richiedente} che trasferirà l'informazione al \textbf{client web richiedente} che potrà ora usare l'IP per connettersi al \textbf{server web richiesto}.
\end{enumerate}

\subsubsection*{DNS Caching}
Per migliorare le prestazioni e ridurre il carico sui server DNS, i risultati delle query DNS vengono memorizzati in una \textbf{cache DNS}. Questa cache può essere presente sia nei \textit{server DNS locali} che nei \textit{client DNS}. Quando un client richiede la traduzione di un hostname, il DNS locale controlla prima la sua cache. Se il record DNS è presente nella cache e non è scaduto (in base al valore \textit{ttl}), il DNS locale restituisce direttamente la risposta al client, evitando di dover effettuare una nuova query nella gerarchia DNS. Questo meccanismo di caching riduce significativamente il tempo di risposta e il traffico di rete.

\subsubsection{Record DNS}
Formato \textbf{RR (Record di risorsa)}: (name, value, type, ttl), dove \textit{ttl} rappresenta il \textit{time-to-live} del record.
4 tipi di type:
\begin{itemize}
  \item \textbf{Type=A}: name = nome host, value = indirizzo \textit{IPv4}.
  \item \textbf{Type=NS}: name = dominio, value = nome del server di competenza, usato per \textit{delegare l'autorità} ad altri server DNS.
  \item \textbf{Type=CNAME (canonical name)}: name = \textit{nome alias} di un nome \textit{nome canonico}, value = \textit{nome canonico}, usato per creare \textit{alias} per gli hostname.
  \item \textbf{Type=MX}: value = nome del server di posta di \textit{name}, name = nome server, usato per \textit{instradare le email} al mail server appropriato.
\end{itemize}

\subsubsection*{Esempio popolazione DB server}

\includegraphics[width=\textwidth]{./img/dbdns.jpg} \\

\bigskip
\textbf{DB del server Radice} \newline
\begin{tabular}{lll}
\textbf{Name} & \textbf{Value} & \textbf{Type} \\
.it           & a.dns.it       & NS            \\
a.dns.it      & 22.4.9.10      & A             
\end{tabular} \\
Il server radice punta al server TLD per il dominio `.it`.

\bigskip
\textbf{DB del server TLD} \newline
\begin{tabular}{lll}
\textbf{Name} & \textbf{Value}     & \textbf{Type} \\
foo.it        & dns.foo.it         & NS            \\
dns.foo.it    & 147.163.2.1        & A             
\end{tabular} \\ 
Il server TLD punta al server autoritativo per il dominio `foo.it`.

\bigskip
\textbf{DB del server DNS autoritativo} \newline
\begin{tabular}{lll}
\textbf{Name} & \textbf{Value}     & \textbf{Type}  \\
www.foo.it    & s1.foo.it          & CNAME          \\
foo.it        & s1.foo.it          & CNAME          \\
s1.foo.it     & 143.163.2.2        & A              \\
foo.it        & s2.foo.it          & MX             \\
s2.foo.it     & 143.163.2.3        & A              
\end{tabular} \\ 
Il server autoritativo contiene gli indirizzi IP degli host nel dominio `foo.it`, incluso il mail server. Notiamo che i record CNAME creano alias per gli hostname e che i record MX specificano il mail server per il dominio.
